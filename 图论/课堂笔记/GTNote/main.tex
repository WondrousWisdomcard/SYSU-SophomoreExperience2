\documentclass{article}
\usepackage[utf8]{inputenc}
\usepackage[utf8]{ctex}
\usepackage{amsmath}
\usepackage{graphicx}

\title{grApH ThEOry nOTE}
\author{mAx zHeNg}

\begin{document}

\maketitle

\tableofcontents

\section{图的基本概念}

\subsection{图的基本概念}

\subsubsection{图的同构}

\texttt{PPT1}

定义1. 设有两个图$G_1 = (V_1,E_1)$和$G_2 = (V_2,E_2)$,若再其顶点集合间存在双射,使得边之间存在如下关系:设$u_1 \leftrightarrow u_2, v_1 \leftrightarrow v_2, u_1,v_1 \in V_1, u_2,v_2 \in V_2; u_1v_1 \in E_1, u_2, v_2 \in E_2$,且两条边的重数相同,则称$G_1$与$G_2$同构,记为$G_1 \cong G_2$。

判断两个图同构,通过建立双映射,罗列边。

\subsubsection{完全图、偶图、补图}

\texttt{PPT2}

定义2. n个顶点的完全图,记为$K_n$,容易得出$m(K_n) = \frac12 n(n-1)$。

定义3. 偶图和完全偶图,后者记为$K_{m,n}$,m,n为两个顶点集的顶点数。

定义4. 补图,对于一个简单图$G = (V,E)$,令集合$E_1 = \{ uv|u \ne v, u,v \in V\} $,则称图$H = (V,E_1 \backslash E)$为图$G$补图,记为$\overline{G}$。

定义5. 自补图:G为自补图当且仅当若图G与图G的补图同构。

定理1. 若G是n阶自补图,则有: $n = 0,1(mod 4)$。证明,因为$m(G) = \frac14n(n-1)$。

定义6. 图的最大度和最小度分别用$\Delta(G) 和 \sigma(G)$表示;奇点和偶点分别表示度数为奇数和偶数的点。

\subsubsection{图的度序列}

\texttt{PPT2}

定义7. 图的拓扑不变量指与图有关的一个数组,他对于与图同构的所有图来说,不会发生改变。

定理2. 非负整数组$(d_1,d_2,...,d_n)$是图的度序列当且仅当$\sum_{i = 1}^n d_i$为偶数。证明,同过构造的方法。

定理3. 非负整数数组$(d_1,d_2,...,d_n)$是简单图图序列的充要条件是$(d_2-1,d_3-1,...,d_{d1+1},d_{d_1+2},...,d_n)$是图序列。

定理4. 一个简单图的n个顶点的度数不能互不相同(鸽笼原理)。

定义8. 设n阶图G的各点的度数取s个不同的非负整数$d_1,d_2,...,d_s$,又设度数为$d_i$的点有$b_i$个($i = 1,2,...,s$),则$\sum_{i=1}^s = n$。故非负整数数组是n的一个划分,称为G的频序列。

定理5. 一个图和它的补图有相同的频序列。

\subsection{图的子图*}

\texttt{PPT3}

定义1. 如果$V(H) \subseteq V(G). E(H) \subseteq E(G)$,且H中边的重数不超过G中对应边的重数,则称H为G的子图,记为$H \subseteq G$。

\quad 当$H \subseteq G, H \ne G$, H是G的真子图,记为$H \subset G$。

定义2. 顶点导出子图: 如果$V' \subseteq V(G)$,则V'为G的顶点子集,以两个端点均在V'中的边集组成图,称为G关于点集V'的点导出子图,记为$G[V']$。

定义3. 边导出子图:如果

\subsection{图的运算}

\texttt{PPT3}

1. 删点:删点要删边,边必须要有端点才能存在;

2. 删边:删边不删点,点可以独立存在;

3. 并运算$\cup$:$G1 \cup G2$:两个图的并,点是点的并,边是边的并

特别地,若两个图不相交(无公共顶点),则称它们地并为直接并,即为$G1+G2$;

4. 交运算$\cap$:两个图的交,点是点的交,边是边的交;

5. 差运算$G1-G2$:从G1中删去G2中的\textbf{边}得到的新图;

6. 对称差运算:$G1 \Delta G2$ = $G1\cup G2 - (G1 \cap G2)$

7. 联运算:联运算是对两个不相交的图G1,G2而做的。$G1 \vee G2$,先做$G1+G2$,之后将G1中每个顶点与G2中每个顶点连接,得到的新图即为$G1 \vee G2$;

8. 积图:积图$G1 \times G2$ ,通过分别求出点集和边集得到最终的积图。

点集:V=$V1 \times V2$, 即V1与V2的笛卡尔积。例如,$V1=\{1,2\};V2=\{3,4,5\}$,则$V=\{(1,3),(1,4),(1,5),(2,3),(2,4),(2,5)\}$。

边集:若顶点$u=(u1,u2),v=(v1,v2)$满足如下两个条件之一,则连接uv;否则,不连接u,v.

条件:$(u1=v1 \cup u2 adj v2)$ 或 $(u2=v2 \cup u1 adj v1)$.

\subsection{图的最短路算法}

\texttt{PPT4}

Dijkstra算法是生成最短路径树的,其思路是:首先把起点到所有点的距离存下来找个最短的,然后松弛一次再找出最短的,所谓的松弛操作就是,遍历一遍看通过刚刚找到的距离最短的点作为中转站会不会更近,如果更近了就更新距离,这样把所有的点找遍之后就存下了起点到其他所有点的最短距离。

1.简单复杂度是$O(n2)$。 Dijkstra 算法最简单的实现方法是用一个链表或者数组来存储所有顶点的集合 Q,所以搜索 Q 中最小元素的运算(Extract-Min(Q))只需要线性搜索 Q 中的所有元素;

2.用堆优化后的时间复杂度:$O((m+n)log n)$

\subsection{图的代数表示*}

\texttt{PPT4}

1. 图的邻接矩阵$a_{ij}$为$v_i$与$v_j$之间的边数。

\subsection{图的邻接谱理论}

\texttt{PPT5}

定义1. 图的邻接矩阵A的特征多项式:$$f(G,\lambda) = |\lambda E - A| = \lambda^n + a_1 \lambda^{n-1} + ... + a_{n-1} \lambda + a_n $$ 其中,A的特征方程有非零解的充要条件是系数行列式为0,即 $| A - \lambda E | = 0$

定理1. 图G的特征多项式的系数:$$a_i = (-1)^i \sum_H \det H \times s(G,H), i = 1,2,...,n$$其中,右边表示对G的所有i阶点导出子图H求和,$s(G,H)$表示G的同构于H的点导出子图的数目。(这么复杂嘛)

例1. 证明简单图G的特征多项式满足:

\quad $a_1 = 0$;

\quad $-a_2 = m(G),m(G)$为图的边数总和;

\quad $–a3$ 是G中含有不同的K3子图(三角形)的个数2倍。

\quad 证明:由矩阵理论:对每个$1 \le i \le n ,(-1)^i a_i $是A(G)的所有i阶主子式之和。A(G)的非零2阶和3阶主子式必有固定形式,
\begin{math}
	\left[
	\begin{smallmatrix}
	0 & 1  \\
	1 & 0 
	\end{smallmatrix}
	\right]
\end{math} = -1和
\begin{math}
	\left[
	\begin{smallmatrix}
	0 & 1 & 1  \\
	1 & 0 & 1 \\
	1 & 1 & 0
	\end{smallmatrix}
	\right]
\end{math} = 2
并分别对应到G的一条边和G中的一个$K_3$子图。

\textbf{定义2. 图的邻接矩阵A(G)的特征多项式的特征值及其重数,称为G的邻接谱。}

例2. 求$K_n$(n阶完全图)的谱。
	
一个完全图的邻接矩阵除了对角线其他的元素都是1,邻接矩阵对应的一个特征向量为全1的元素,它对应的特征值$\lambda = n - 1$。

一个图有两个不连通的完全子图(分别是n阶和m阶),邻接矩阵是一个对角分块矩阵,可以看出它有两个特征向量,[0,...0,1...1]和[1,...,1,0,...,0],他们是正交的且特征向量分别是n-1和m-1。

所以我们可以从这样的特征向量和特征值推广可以对图得到大概的分类,分类依据是节点聚集的情况。

$|\lambda E - A(K_n)| $ = 
\begin{math}
	\left[
	\begin{smallmatrix}
	\lambda & -1 & -1  \\
	-1 & \lambda & -1 \\
	-1 & -1 & \lambda
	\end{smallmatrix}
	\right]
\end{math} = 
\begin{math}
	\left[
	\begin{smallmatrix}
	\lambda-n+1 & -1 & -1  \\
	\lambda-n+1 & \lambda & -1 \\
	\lambda-n+1 & -1 & \lambda
	\end{smallmatrix}
	\right]
\end{math} (第一列减去其他列的和)=
$(\lambda - n + 1)$\begin{math}
	\left[
	\begin{smallmatrix}
	1 & -1 & -1  \\
	1 & \lambda & -1 \\
	1 & -1 & \lambda
	\end{smallmatrix}
	\right]
\end{math} 
$ = (\lambda - n + 1)(\lambda + 1)^{n-1} $
又因为矩阵线性相关,行列式为0,故求得两个特征值 -1, n-1,它们的重数分别是n-1和1。

因此有该特征谱可以表示为:
$Spec(K_n) = $\begin{math}
	\left[
	\begin{smallmatrix}
	-1 & n-1 \\
	n-1 & 1
	\end{smallmatrix}
	\right]
\end{math} (第一列为特征值,第二列为重数)

例3:若两个非同构的图具有相同的谱,则称他们为同谱图。

例4:简单图的谱为 Spec(G) = 
\begin{math}
	\left[
	\begin{smallmatrix}
	\lambda_1 & \lambda_2 & ... & \lambda_s \\
	m_1 & m_2 & ... & m_s
	\end{smallmatrix}
	\right]
	\end{math} 有: $\sum_{i=1}^{s}m_i\lambda_i^2 = 2m(G)$ 其中m(G)表示简单图的边数,$m_i$表示重数。

例5:设$\lambda$是简单图G的任意特征值,则有$|\lambda| \le \sqrt{\frac{2m(n-1)}{n}}$。

\subsection{图的邻接代数}

\texttt{PPT5}

定义3. 设A是无环图的邻接矩阵,C是一个数域,称

$$\Lambda(G) = \{ a_0E + a_1A + ... + a_k A^k | a_i \in C, k \in Z^+ \}$$

(上面定义了一个集合)对于$\textbf{矩阵加法和数与矩阵的乘法}$来说作成数域C上的向量空间(性质:非空,关于加法和数乘封闭,八条(单位元,零元,负元,分配律,交换律,等等))称该空间为图G的邻接代数。

定理1. 令G为n阶连通无环图,则:$$ d(G) + 1 \le dim \Lambda (G) \le n$$ 其中,d(G)为图的半径:任意两点路径的最大值。

定理1的应用:对于完全图,有$2 \le dim \Lambda (K_n) \le n$

例6: n阶连通无环图G的邻接矩阵的不同特征值个数S满足$ d(G) + 1 \le S \le n$

由矩阵理论,非负对称矩阵的不同特征值个数等于其最小多项式的次数(这是什么东西),而后者等于G的邻接代数的维数。

\subsection{图空间简介}

\texttt{PPT5}

定理2. 集合$M = {G_1,G_2...G_N | G_i为图G的生成子图,N = 2^m}$,对于图的对称差运算(两个集合的并集减去交集)和数乘运算:$0 \cdot G_i = \Phi , 1 \cdot G_i = G_i$,来作为数域F = {0,1} 上的m维向量空间。

(基底:集合内的东西都可以表示成这组基底的线性组合。)

\subsection{极图理论简介}

\texttt{PPT6}

\subsubsection{l部图}

l部图是二部图的推广。

定义1. \textbf{l部图},若简单图G的点集V有一个划分: $$ V = {_{i=1}^l V_i. V_i \bigcap V_j = \Phi, i \ne j }$$,对于所有的$V_i$非空,且集合内的元素互不邻接。

定义2. \textbf{完全l部图},任意部$V_i$中的任意顶点于其他各部的所有顶点邻接,记$K_{n_1,n-2,...,n_l}$。

$$|V| = \sum_i^l n_i, |G| = \sum_{i < j}^l n_i\cdot n_j$$

定义3. n结点完全l部图中,$n = k \times I + r, 0 \le r < I, |V_1| = |V_2| = ... = |V_r| = k + 1$, 剩下的$|V_i| = k$,则G为称\textbf{n阶完全l几乎等部图},记为$T_{l,n}$ (V是点集),$r = 0$,即$V_i$全部相等,则称\textbf{完全l等部图}。

定理1. \textbf{连通}的偶图的二部划分是唯一的。

定理2. n阶完全偶图$K_{n_1,n_2}$的边数为$m = n_1 \cdot n_2$,且有:$m \le \lfloor \frac{n^2}{4} \rfloor $

定理3. n阶l部图G有最多边数的充要条件是$G \cong T_{l,n}$,$\cong$指同构,$T_{l,n}$是\textbf{n阶完全l几乎等部图}

\subsubsection{托兰定理}

定义4. 设G和H是两个n阶图,称\textbf{G度弱于H},如果存在双射$\mu :V(G) \to V(H)$,使得$$\forall v \in V(G), d_G(V) \le d_H(\mu(v))$$

若G度弱于H,则一定有$m(G) \le m(H)$

定理4. 若n阶简单图G中不包含$K_{l+1}$(l+1阶完全图),则G度弱于某个完全l部图H,且若G具有和H相同的度序列,则:$G \cong H$(同构)。

N(u)是u的邻接顶点集,$V$联运算:二部图连法。

定理5. 托兰定理:若G是简单图,并且不包含$K_{l+1}$,则:
$$m(G) \le m(T_{l,n})$$此外,仅当$G \cong T_{l,n}$时,有$m(G) = m(T_{l,n})$。

结论:

设m(n,h)表示n阶单图中不包含子图的最多边数,则:
\quad 1. $m(n,K_3) = \lfloor \frac{n^2}{4} \rfloor$

\quad 2. 数学建模举例:

问题转化成:
任意两个人之间的距离不超过G米的条件下,距离大于H米的人数对最多能达到多少。
图模型:
每个士兵用一个点表示,两点连线当且仅两个人距离大于H。

在给定的长度中可以证明部数不得大于3,遂在$T_{3,n}$下有最多长度大于H的边,故可以得到一个散步图,每个子图的半径比较小,每个补圆圈的最小距离大于H。

\subsection{交图和团图}

\texttt{PPT6}

定义1.  \textbf{交图}:设S是一个集合,$F = \{S_1,S_2,...,S_n\}$是S的不同的非空子集的一个非空族,他们的并是$S$,集族F的交图,记为$\Omega(F)$,定义为:$V(\Omega(F)) = F$,当$i \ne j$且$S_i \cap S_j \ne \Phi$时,$S_i$和$S_j$邻接。

定理1. 每个图都可以是一个交图,证明构造,子集是每个点的两条边。

定义2. G是S上的交图,如果S的基数(集合元素个数)最小,称S的奇数位图G的交数,记为$v(G)$.

定理3. 
\quad(2). 若G有m条边,$n_0$个孤立点,无$K_2$支,则$v(G) \le m + n_0$。

\quad(3). 若G为连通图(阶数大于3),G中没有三角形当且仅当$v(G) = m$。

定义3. 简单图G的一个\textbf{团}(Clique)指V中的一个子集$V_1$,使得$G[V_1]$($V_1$的生成子图)是完全图。

定义4. 一个给定的\textbf{图G的团图}是G的团的族的交图。

定理4. 一个图G是一个团图当且仅当它含有完全子图的一个族F,他们的并是G,且如果F的某个子族F'中的每一对完全子图的交非空,则F'的所有元素的交就非空。

\section{树}

\subsection{树的基本概念}

\texttt{PPT7}

定义1. 树是连通的无圈图,若不连通则为森林。

定理1. 树和森林都是偶图,也都是简单图。

定理2. 每棵非平凡树至少有两片树叶。

定理3. G是树当且仅当任意两点有且仅有唯一路径。

定理4. $m = n - 1$,m为树的边数,n为树的点数,做归纳证明。

推论1. 有k分支的森林有,n-k条边。

定理5. n阶连通图的边数至少为你n-1,证明技巧,有无1度顶点。

定理6. 任意树的两个不邻接的点之间添加一条边可以得到唯一的圈。

定理7. G是一棵树且最大度$\delta > k$,则G至少有k个一度顶点。

定理8. 正整数序列$s = (d_1,d_2,...,d_n)$,满足$d_1 \le d_2 \le ... \le d_n$,且$\sum d_i = 2(n-1)$,则存在一个树度数序列为S

\subsection{树的形心}

\texttt{PPT7}

回顾1. 图的顶点的离心率:$e(v) = max\{d(u,v)|u \in V(G)\}$;

\quad 图的半径:$r(G) = min\{e(v) | v \in G\}$;

\quad 图的直径:$d(G) = max\{e(v) | v \in G\}$;

\quad 图的中心点:离心率等于半径的点;

\quad 图的中心:图的中心点的集合1。

定理9. 每棵树的中心是有一个点或者两个相邻的点组成(证明:每次删去一层叶子)。

定义2. 树的形心:设u是树T的任意一个顶点,T在\textbf{顶点u的分支}是包含u作为一个叶子节点的极大子树,其分支数为u的度数;树T在u分支中边的最大数目称为\textbf{点u的权};树T中权值最小的点是它的一个\textbf{形心点};(树T中权值最大的点是T的所有叶子结点,其权为T图的边数)所有形心点的集合称为树的\textbf{形心}。

定理10. 每棵树的形心是有一个点或者两个相邻的点组成(证明:每次删去一层叶子)。

\subsection{生成树的概念和性质}

\texttt{PPT8}

\subsubsection{生成树的概念和性质}

定义1. 若n阶图G的n阶(边)生成子图T为树,则称T为G的生成树(生成树一般不唯一)。

定理1. 每个连通图必包含一棵生成树(证明:破圈法)。

\subsubsection{生成树的计数问题}

1. Cayley凯莱递推计数法

定义2. 图G的边e称为被收缩,是指删掉$e$后,把$e$的两个端点重合,如此得到的图记为$G.e$。

定理2. (凯莱定理) 设e是G的一条边,则$\tau(G) = \tau(G-e) + \tau(G.e)$($\tau$是生成数的个数,证明思路:$\tau(G-e)$是$G$中不包含e的生成数的个数,:$\tau(G.e)$是$G$中包含生成数的个数)。

算法1. 由凯莱定理递归,算法复杂度$O(2^m)$,(但是你需要思考停机条件是什么)。

2. 关联矩阵计数法

定义3. $n\times m$矩阵的一个阶数为$min\{n,m\}$的子方阵,称为他的一个主子阵,主子阵的行列式为主子行列式;一个矩阵有$C_{max(n,m)}^{min(n,m)}$个主子阵。

定理3. 设$A_m$(n-1阶方阵)是连通图G的基本关联矩阵的主子阵$A_f$($n-1\times m$的矩阵),则$A_m$非奇异的充要条件是相应于$A_m$的列的那些边构成的G的一棵生成树。

Background1. 图的基本关联矩阵:在关联矩阵中划去任意一点所对应的行,得到的矩阵,其秩为$n-1$。

Background2. 非奇异矩阵就是满秩矩阵,也叫做可逆矩阵。

算法2. 

\quad\quad (1). 给出关联矩阵,进一步写出一个基本关联矩阵;
    
\quad\quad (2). 找出基本关联矩阵的非奇异主子阵,对于每一个主子阵,画出其相应生成树。

定理4. (矩阵树定理) 设G是顶点集合为$V(G) = {v_1,v_2,...,v_n}$的图,设$A = (a_{i j})$是G的邻接矩阵, $C = (c_{i j})$是n阶方阵,其中 $C_{i j} = (i == j)  ?  (d(v_i)) : (-a_{i j})$。\textbf{G的生成树的棵数是C的任意一个元素的代数余子式}。(矩阵C又称为图的拉普拉斯矩阵,又可以定义为$C = D(G)-A(G)$,$D(G)$是图的对角度数矩阵,对角未度数,其余元素为0,$A(G)$是邻接矩阵)

Background2. 代数余子式:在n阶行列式中,把元素$a_{o e}$所在的第o行和第e列划去后,留下来的n-1阶行列式叫做元素$a_{o e}$的余子式,记作$M_{o e}$,将$M_{o e}$再乘以$(-1)^{o+e}$记为$A_{o e}$,$A_{o e}$叫做元素$a_{o e}$的代数余子式。

定理5. 完全图的生成树个数为$n^{n-2}$,可以用矩阵树定理直接算出来。

\subsubsection{回路系统简介}

定义4. 设T是连通图G中的一颗生成树,把属于G但不属于T的边称为G关于T的连枝,T中的边称为T关于G的树枝。

定义5. 假设T是连通图G的一棵生成树,由G的对应于T的一条连枝于T中的树枝构成唯一的圈C,称为G关于T的一个基本圈,或者基本回路。若G是(n,m)连通图,把G对应于T的m-n+1个基本回路称为G对应于T的基本回路组。记为$C_f$

定理6. 设T是连通图G的一棵生成树,有$m+n-1$个基本回路$C_i$,定义:$1\cdot G_i = G_i, 0\cdot G_i = \Phi$,其中$G_i$是G的回路(与基本回路区分开)。则G的回路组作成的集合对于该数乘和图的对称差运算来说作成数域$F = \{0,1\}$ 上 $m+n-1$ 维的向量空间。

需要注意的是基本回路时基于一棵生成树而产生的,所以图不仅只有一种基本回路组。

\subsection{最小生成树算法}

\texttt{PPT9}

最小生成树:原图的极小连通子图,且包含原图中的所有 n  个结点,并且有保持图连通的最少的边。

1. Kruskal算法

思路:从G中的最小边开始,进行避圈式扩张。复杂度:$O(m \cdot log(m))$

2. 破圈法(管梅谷)

思路:从赋权图G的任意圈开始,去掉该圈中权值最大的一条边,称为破圈。不断破圈,直到G中没有圈为止,最后剩下的G的子图为G的最小生成树。

3. Prim算法

从G中的任意点开始,选择关联的权重最小且不生成圈的边添加,直到得到最小生成树。简单实现的复杂度:$O(n^2)$,堆优化复杂度:$O(n \cdot log(n))$(不确定)

\textbf{区分:Dijkstra算法是生成最短路径树的},其思路是:首先把起点到所有点的距离存下来找个最短的,然后松弛一次再找出最短的,所谓的松弛操作就是,遍历一遍看通过刚刚找到的距离最短的点作为中转站会不会更近,如果更近了就更新距离,这样把所有的点找遍之后就存下了起点到其他所有点的最短距离。(Prim算法的代码实现和Dijkstra很像)

例5. 连通图G的树图是这样的图:它的顶点是G的生成树$T_1,T_2,...$,它们相连仅当它们恰好存在n-2条公共边,证明任何连通图的树图都是连通图(证明方法:一步一步构造出任意$T_i, T_j$的一条路)。

\subsection{计算机中的树}

定义2:一棵树T,如果每条边都有一个方向,称这种树为有向树。对于T的顶点v来说,以点v为终点的边数称为点v的入度,以点v为起点的边数称为点v的出度。入度与出度之和称为点v的度。

定义3:一棵非平凡的有向树T,如果恰有一个顶点的入度为0,而其余所有顶点的入度为1,这样的有向树称为根树。其中\textbf{入度为0的点称为树根},\textbf{出度为0的点称为树叶},\textbf{入度为1,出度大于0的点称为内点}。又将内点和树根统称为\textbf{分支点}。

定义4:对于根树T,顶点v到树根的距离称为点v的层数;所有顶点中的层数的最大者称为根树T的树高。

定义5:对于根树T,若规定了每层顶点的访问次序,这样的根树称为有序
树。

定义6:对于根树T,由点v及其v的后代导出的子图,称为根树的子根树。

定义7:对于根树T,若每个分支点至多m个儿子,称该根树为m元根树;若每个分支点恰有m个儿子,称它为完全m元树。

定理2:在完全m元树T中,若树叶数为t , 分支点数为i , 则:$(m-1)i=t-1$。

定义8 设T是一棵二元树,若对所有t片树叶赋权值$w_i(1\le i \le t)$,且权值为$w_i$的树叶层数为$L(w_i)$,称:
$$ W(T) = \sum_{i = 1}^{t}w_i\cdot L(w_i)$$ 
为该赋权二元树的权。而在所有赋权为$w_i$的二元树中;W(T)最小的二元树称为\textbf{最优二元树}(可以通过哈夫曼算法得到)。

\section{图的连通性}

\subsection{割边、割点、块}

\texttt{PPT10}

定义1 割边:边$e$为图G的一条割边,如果 $\omega(G-e) \le \omega(G)$,$\omega(G)$表示极大连通子图的数量。

定理1:(充要条件)边 $e$ 是图G的割边当且仅当 $e$ 不在G的任何圈中。

推论1: $e$ 为连通图G的一条边,如果 $e$ 含于G的某圈中,则 $G-e$ 连通。

例1: 求证: (1) 若G的每个顶点的度数均为偶数,则G没有割边; (2) 若G为k正则二部图($k\le2$),则G无割边。

证明: (1)若不然,设$e=uv$ 为G的割边。则$G-e$的含有顶点u(或v)的那个分支
中点u(或v)的度数为奇,而其余点的度数为偶数,与\textbf{握手定理}推论相矛盾。

定义2 \textbf{连通图的秩}:一个具有n个顶点的连通图G,定义$n-1$为该连通图的\textbf{秩};具有p个分支的图的秩定义为$n-p$。记为$R(G)$。

定义3 \textbf{边割集}:设S是连通图G的一个边子集,如果:(1) $R (G-S) = n-2;$ (2) 对S的任一真子集$S_1$,有$R(G-S_1) = n-1$。称S为G的一个边割集,简称G的一个边割(任意一个图都会有边割集)。

定义4 \textbf{关联集}:在G中,与顶点v关联的边的集合,称为v的关联集,记为:$S (v)$。

定义5 \textbf{断集}:在G中,如果$V=V1 \cup V2,V1 \cap V2=\Phi$,$E_1$是G中端点分属于$V_1$与$V_2$的G的边子集,称$E_1$是G的一个断集(割集、关联集是断集,但逆不一定)。

定理2: 任意一个断集均是若干关联集的环和($E_1 \oplus E_2 =  (E_1 - E_2) \cup  (E_2 - E_1) = E_1 \cup E_2 - E1 \cap E_2$)。

定理3: 连通图G的断集的集合作成子图空间的一个子空间,其维数为n-1。该空间称为图的断集空间。(其基为n-1个线性无关的关联集)

定义6 \textbf{基本割集}:定义6 设G是连通图,T是G的一棵生成树。如果G的一个割集S恰好包含T的一条树枝,称S是G的对于T的一个基本割集。

定理4:连通图G的断集均可表为G的对应于某生成树T的基本割集的环和。

定理5:连通图G对应于某生成树T的基本割集的个数为n-1,它们作成断集空间的一组基。

(我们在子图空间基础上,先后引进了图的回路空间和断集空间,它们都是子图空间的子空间)

定义7 \textbf{割点}:在G中,如果E(G)可以划分为两个非空子集E1与E2,使$G[E_1]$和$G[E_2]$以点v为公共顶点,称v为G的一个割点。

定理6:G无环且非平凡,则v是G的割点,当且仅当$\omega(G-v) > \omega(G)$,$\omega(G)$表示极大连通子图的数量。

定理7:v是树T的顶点,则v是割点,当且仅当v是树的分支点。

定理8:设v是无环连通图G的一个顶点,则v是G的割点,当且仅当$V(G-v)$可以划分为两个非空子集V1与V2,使得对任意$x \in V1, y \in V2$, 点v在每一条x到y的路上。

例5 求证:无环(无自环)非平凡连通图至少有两个非割点(思路:而非平凡生成树至少两片树叶)。

例6 求证:恰有两个非割点的连通单图是一条路(思路:G的任意生成树为)。

例7 求证:若v是单图G的割点,则它不是G的补图的割点。

定义8 \textbf{块}:没有割点的连通图称为是一个块图,简称块;G的一个子图B称为是G的一个块,如果(1). 它本身是块;(2). 若没有真包含B的G的块存在。

定理9:若$|V(G)|\ge 3$,则G是块,当且仅当G无环且任意两顶点位于同一圈上(必要性:归纳法,找圈;充分性:反证)。

定理10:点v是图G的割点当且仅当v至少属于G的两个不同的块。

(该定理揭示了图中的块与图中割点的内在联系:不同块的公共点一定是图的割点。也就是说,图的块可以按割点进行寻找。所以,该定理的意义在于:可以得到寻找图中全部块的算法。)

定义9 \textbf{块割点树}:设G是非平凡连通图。$B_1, B_2 ,…, B_k$是G的全部块,而$v_1,v_2,…,v_t$是G的全部割点。构造G的块割点树bc(G):\textbf{它的顶点是G的块和割点,连线只在块割点之间进行,一个块和一个割点连线,当且仅当该割点是该块的一个顶点。}

\subsection{图的连通度与敏格尔定理}

\texttt{PPT11}

\subsubsection{点、边连通度}

定义1 \textbf{顶点割}:给定连通图G,设$V' \subseteq G$,若$G - V'$不连通,称V'为G的一个点割集,含有k个顶点的点割集称为k顶点割。G中点数最少的顶点割称为最小顶点割。

定义2 \textbf{点连通度}:在G中,若存在顶点割,称G的最小顶点割的顶点数称为G的点连通度;否则称n-1为其点连通度. G的点连通度记为$\kappa(G)$ , 简记为$\kappa$;若G不连通,$\kappa(G) = 0$ .

定义3 \textbf{边连通度}:在G中,最小边割集所含边数称为G的边连通度。边连通度记为$\lambda(G)$ 。若G不连通或G是平凡图,则定义$\lambda(G) =0$。

定义4 在G中,若$\kappa(G) \ge k$ , 称G是k连通的;若$\lambda(G) \ge k$,称G是k边连通的。

定理1 (惠特尼1932) :对任意图G,有$\kappa(G) \le \lambda(G) \le \delta(G)$。(后者是图的最小度,后半部分证明:最小度顶点的关联集作成G的断集,前半部分的证明:快逃)

定理2 设G是(n, m)连通图,则$\kappa(G) \le \lfloor \frac{2m}{n} \rfloor$。

定义 \textbf{哈拉里图}:存在一个(n, m) 图G,使得$\kappa(G)  = \lfloor \frac{2m}{n} \rfloor$,哈拉里构造了\textbf{连通度是k},边数为$m  = \lceil \frac{nk}{2} \rceil$的图$H_{k,n}$,称为哈拉里图。

定理2 设G是(n, m)单图,若$\delta(G)  = \lfloor \frac{n}{2} \rfloor$,则G连通。

定理3 设G是n阶简单图,若对任意正整数$k < n$ ,有:$\delta(G) \ge \frac{n+k-2}{2}$,则G是k连通的。

定理4 设G是n阶简单图,若$\delta (G) \ge \lfloor \frac{n}{2} \rfloor$,则有$\lambda (G) = \delta(G)$。

\subsubsection{坚韧度}
定义1 \textbf{坚韧度}:用C (G)表示图G的全体点割集构成的集合,非平凡非完全图的坚韧度,记作$\tau(G)$,定义为:
$$\tau(G) = min\{ \frac{|S|}{\omega(G-S)}|S\in C(G)\}$$
$\omega(G)$表示极大连通子图的数量。
定义2 设G是一个非完全$n (n\ge 3)$阶连通图,$S* \in C (G)$。若S*满足:
$$\tau(G) = \frac{|S*|}{\omega(G-S*)}$$
称S*是G的坚韧集。

易知:坚韧集是那些顶点数尽可能少,但产生的分支数尽可能多的点割集,同时,坚韧集不唯一。因此,坚韧度可以作为网络容错性参数的度量。仿照点坚韧度,可以定义边坚韧度。

\subsubsection{图的核度}

定义3 设G是一个非平凡连通图,则称:
$h(G) = max\{\omega(G-S) - |S||S \in C(G)\}$
为图的核度。若S*满足:
$ℎ(G) = \omega(G − S *) − |S*|$,称S*为图的核,一般地,核度越小,连通程度越高。

\subsubsection{敏格尔定理}

\texttt{PPT12}

定义1: S分离u和v

定理1 \textbf{敏格尔定理}:

\quad(1)假设u和v是两个不相邻的顶点,则\textbf{G中分离u和v的最少点数等于独立的(x,y)路的最大数目。}(独立指没有相邻的内点)

\quad(2)假设u和v是两个不相邻的顶点,则\textbf{G中分离x和y的最少边数等于G中边不重合的(x,y)路的最大数目。}(边不同,比如两条路可以交叉)

直观意义是在每一条路上设置一个关卡(点/边)。

定理2 \textbf{惠特尼(1932)}  一个非平凡图G是k连通的,当且仅当G的任意两个顶点间至少存在k条内点不相交的(u,v)路。(k连通的定义:分离中两个不相邻的顶点至少需要k个顶点,若$\tau(G) \ge k$,称G是k连通的)(证明用敏格尔定理)

定理3 一个非平凡的图G是$k (k\ge2)$边连通的,当且仅当G的任意两个顶点间至少存在k条边不重的(u ,v)路。

推论 对于一个阶数至少为3的无环图G,下面命题等价:

\quad 1. G是2连通的

\quad 2. G中任意两点位于同一个圈上

\quad 3.  G无孤立点,且任意两条边在同一个圈上


\subsection{图的宽直径简介}

\subsubsection{问题背景}

\textbf{直径}:$max\{e(v) | v \in G\}$能够刻画网络中的通信延迟。

定理1 强连通有向图直径和最大度的关系:$n \le 2, \Delta$为最大度,则...


定理2 连通无向图直径和最大度的关系:...

定理3 连通无向图直径和最小度$\delta $的关系:$d(G) \le \frac{3n}{\delta + 1}$,$n$为阶数。

定理4 连通无向图边数和直径和阶数的关系:$E(G) \le k + \frac{1}{2}(n-k+4)(n-k+1)$。

定理5 n级超立方网络的直径为n。

定义 \textbf{平均距离}:$\mu(G) = \frac{1}{n(n-1)}\sum_{u,v\in V(G)}d(u,v)$。

定理6 G是n阶连通无向图,则$\mu(G) \le \frac{n}{\delta+1}+2$。

定理7 G(n,m)图:若G为无向图,满足最小度$\sigma = $,若G为有向图,满足最小度$\sigma = $。

\subsubsection{宽直径相关概念}

定义1 \textbf{路族,路族的宽度,路族的长度}:设$x, y \in V(G)$, $C_w (x, y)$表示G中w条内点不交路的路族,w称为路族的宽度, $C_w (x, y)$中最长路的路长成为该路族的长度,记为:$l (C_w (x, y))$。

定义2 \textbf{宽距离}:设$x, y \in V(G)$, 定义x与y间所有宽度为w的路族长度的最小值$d_w(x,y)$为x与y间w宽距离,即:$d
_w(x,y) = min\{l(C_w(u,v)):\forall C_w(u,v)\}$。

定义3 \textbf{宽直径}:设G是\textbf{w连通}的,G的所有点对间的\textbf{w宽距离的最大值},称为G的w宽直径,记为$d_w (G)$。即:$d_w(G) = max\{d_w(x,y):x,y \in V(G)\}$

\section{欧拉图与哈密尔顿图}

\subsection{欧拉图极其性质}

定义1 \textbf{欧拉图}:对于连通图G,如果G中存在经过每条边的闭迹,则称G为欧拉图,简称G为E图。欧拉闭迹又称为欧拉环游,或欧拉回路

定理1 下列陈述对于非平凡连通图G是等价的:
(1) G是欧拉图;
(2) G的顶点度数为偶数;
(3) G的边集合能划分为圈。

推论1 连通图G是欧拉图当且仅当G的顶点度数为偶。
推论2 连通非欧拉图G存在欧拉迹当且仅当G中只有两个顶点度数为奇数。

\subsection{Fleury算法}

方法是尽可能避割边行走。(割边指未选边子图的割边)

例4:证明若G有$2k>0$个奇数顶点,则存在k条边不重的迹$Q_1, Q_2, ... Q_k$,使得:$E(G) = E(Q_1) \cup E(Q_2) \cup ... \cup E(Q_k)$ (证明思路:加边扩展到可以欧拉环游,然后再把加的边删掉。)

例5 设G是非平凡的欧拉图,且$v \in V(G)$。 证明:G的每条具有起点v的都能扩展成G的欧拉环游当且仅当G-v是森林。(不懂)

\subsection{中国邮路问题}

邮递员派信的街道是边赋权连通图。从邮局出发,每条街道\textbf{至少}行走一次,再回邮局。如何行走,使其行走的环游路程最短?

定理2 若W是包含图G的每条边至少一次的闭途径,则W具有最小权值当且仅当下列两个条件被满足:

(1) G的每条边在W中最多重复一次;

(2) 对于G的每个圈上的边来说,在W中重复的边的总权值不超过该圈非重复边总权值。

(证明应该好好看看)

例6 如果一个非负权的边赋权图G中只有两个奇度顶点u与v,设计一个求其
最优欧拉环游的算法。

(1) 在u与v间求出一条最短路P; (最短路算法)

(2) 在最短路P上,给每条边添加一条重复边得G的欧拉母图G*;

(3) 在G的欧拉母图G* 中用Fleury算法求出一条欧拉环游。

\subsection{哈密尔顿图的概念}

定义1 \textbf{哈密尔顿图}

定义2 \textbf{哈密尔顿路}

定理1 哈密尔顿图必要条件:若G为H图,则对V(G)的任意非空子集S,有$\omega (G-S) \le |S|$, 其中$\omega$表示图的连通度。

定理2 哈密尔顿图充分条件:对于$n>3$的简单图G,若$\delta(G) \ge \frac{n}{2}$,($\delta$指图的最小度)则是哈密尔顿图。

定理3 哈密尔顿图充分条件:对于$n>3$的简单图G,任意不相邻的顶点有$d(u)+d(v) \ge n$,则G是哈密尔顿图。

引理1 对于简单图G,如果G中存在两个不相邻的顶点u和v,满足$d(u) + d(v) \ge n$,那么,G是哈密尔顿图当且仅当$G+uv$是哈密尔顿图。

定义3 \textbf{闭图}: n阶简单图中,若对$d(u) + d(v) \ge n$的任意一对顶点$u,v$,均有u邻接v,则称G是一个闭图。

引理2 若$G_1,G_2$是同一个点集V的两个闭图,则$G = G_1 \cap G_2$也是闭图。

定义4 闭包 称$\overline G$ 是图G的闭包,如果它是原图的极小闭图。

引理3 一个图的闭包是唯一的。

定理4 闭包定理 图G是H图当且仅当它的闭包是H图。

推论1 G是$n > 3$的简单图:
(1)若$\delta(G) \ge \frac{n}{2}$,则G是H图。(Dirac定理)
(2)若对于G的任意不相邻顶点u和v,都有$d(u) + d(v) \ge n$,则G是H图。(Ore定理)

定理5 度序列判定法 Chvatal定理 简单图G的(n > 3)

范定理:若图中每对距离为2的点中至少有一点的度数至少是图的点数的一半,则图是哈密尔顿图。

\subsection{TSP问题}
一个商人要到若干个城市,每座城市只经过一次,如何安排路线使得总行程最短。

\subsubsection{边交换近似算法}
(1). 找到一个哈密尔顿圈$C_0 = v_1,v_2,...,v_n,v_1$
(2). 若存在$w(v_i,v_i+1) + w(v_j,v_j+1) \ge w(v_i,v_j) + w(v_i+1,v_j+1)$,则把C修改为$C_1 = v_1,v_2,...,v_i,v_j,...v_{i+1},v_{j+1}...v_n,v_1$

\subsubsection{最优H圈的下界}
可以通过如下方法求出最优H圈的下界:
1. 在图G中删去任意一点得到$G_1$
2. 对子图$G_1$求出一棵最小生成树T;
3. 在v的关联边中选出两条权值最小者$e_1,e_2$
若H是最优圈,则:$W(H) \ge W(T) + W(e_1) + W(e_2)$。

\section{图的匹配问题}

\subsection{图的匹配与贝尔热定理}

\textbf{定义1 匹配、对集、独立集M}: 若M是G的一个边子集,如果M中的每条边的两个端点只有该条边与这两个端点相连,则M称为一个匹配或对集或边独立集。

\textbf{定义2 M饱和点与非饱和点}:与匹配M的任意边关联的点称为M饱和点,否则称为非饱和点。

\textbf{定义3 最大匹配和完美匹配}: 如果M是图G的包含边数最多的匹配,称M是G的一个最大匹配。特别是,若最大匹配饱和了G中的所有顶点,称为G的一个完美匹配。一个图不一定存在完美匹配。

\textbf{定义4 M交错路和M可扩路}:M是图G的一个匹配,G是一条M饱和边和非饱和边交错形成的路,称为G中的一条交错路。特别的,如果M交错路的起点和终点都是M的非饱和点,则称M交错路为\textbf{M可扩路}。

\textbf{定理1 贝尔热定理}:G的匹配是M的最大匹配,当且仅当G不包含M可扩路。(它提供了一种扩充图的匹配的思路)

\subsection{偶图的匹配问题}

匹配是否存在,如何求出匹配?不同完美匹配的个数?

\textbf{定理2 Hall定理}:G是(X,Y)二部图,G存在饱和X的每个顶点的匹配(从X到Y的匹配)的充要条件是:对于任意S属于X,有|N(S)|大于等于|S|,其中N(S)是S的邻集,即图G中与所有S相邻接的顶点集合。(显然,S在X当中,所以N(S)在Y当中,都是点集)(若X饱和Y,Y饱和X不一定,因为两个点集的点数未必相同,完美匹配需要划分两边的点数相同)(充分性证明:用可扩路)。

推论:若G是正则偶图,则G存在完美匹配。(正则图是指各顶点的度均相同的无向简单图)

例子1:每个K立方体都有完美匹配(k大于等于2)

例子2:求$K_{2n}, K_{n,n}$完全图的完美匹配个数:

例子3:树中至多存在一个完美匹配。(反证法:多完美匹配存在圈,与树矛盾)

\subsection{点覆盖与哥尼定理}

\textbf{定义5 点覆盖、最小点覆盖和覆盖数}:G的一个顶点子集K称为G的一个点覆盖,如果G的每一条边都至少存在一个端点在K中。G中包含点数最少的点覆盖称为最小点覆盖,其包含的点数为G的覆盖数,记为$\alpha(G)$。

\textbf{定理3 匹配与覆盖关系}:设M是G的匹配,K是G的覆盖,若|M| = |K|,则M是G的最大匹配,K是G的最小覆盖。(M是边数,K是点数)

至于为什么一个图的匹配数小于等于覆盖数,可以这样理解,对于图的匹配的每一条边至少对应一个覆盖点,对于非匹配部分可能还需要覆盖点,因此覆盖点数总是大于等于匹配边数。

定理4 哥尼定理:在偶图中,最大匹配的边数等于最小覆盖的顶点数。

例4: 矩阵的一行或一列成为矩阵的一条线,在布尔矩阵中,证明包含所有”1“的线的最小数目,等于具有性质”任意两个1都不在同一条线上的1的最大数目“。

定理5 托特定理:G有完美匹配当且仅当对V(G)的任意非空真子集S,有$o(G-S) \le |S|$,前者表示奇分支的数目(连通分支的顶点总数为奇数称为奇分支)

推论 皮得森定理:没有割边的三正则图存在完美匹配。

\subsection{匈牙利算法}

\subsubsection{在偶图中寻找完美匹配}

问题:偶图$G=(X,Y),|X|=|Y|$,在G中求一完美匹配M

基本思想:从任一初始匹配$M_0$出发,通过寻求$M_0$的一条可扩路$P$,令$M_1 = M_0 \Delta E(P)$,得到比$M_0$大的匹配$M_1$,迭代。

定义1 交错树:$G = (X,Y)$,M是G的匹配,u是M的非饱和点,H是G扎根于点u的M交错树,如果:
(1) $u \in V(H)$
(2)$\forall v \in V(H),(u,v)$路是一个交错路。

直接看算法:
\textbf{偶图的完美匹配算法:匈牙利算法}
令M是初始匹配。H是扎根于M的非饱和点u的交错树,令$S = V(H) \cap X, T = V(H) \cap Y$。
(a) 若M饱和X的所有顶点,停止。否则,设$u$为$X$中M非饱和顶点,置$S = \{u\}, T = \{ \Phi\}$;
(b) 若$N(S) = T$,则G中不存在完美匹配,否则设$y \in N(S) - T$。($N(S)$
指S的临接点集)
(c) 若y为M饱和点,且存在$z$,$yz \in M$,置$S = S \cup \{z\}, T = T \cup \{y\}$,转(b)。否则,设P为M可扩路,置$M_1 = M \Delta E(P)$,转(a)。($\Delta$是运算符)

算法性质:复杂度为$O(|X|^3)$
(1) 最多循环$|X|$次
(2) 初始匹配最多扩张$|X|$次
(3) 生长树最多生长$2|X|-1$次

\subsubsection{在偶图中寻找最大匹配}

问题:在一般偶图上求最大匹配M

分析:使用匈牙利算法求完美匹配时,当在扎根于M 非饱和点u的交错树上 有|N(S)|<|S|时,由Hall定理,算法停止。要求出最大匹配,应该继续检查X-S是否为空,如果不为空,则检查是否在其上有M非饱和点。一直到所有M非饱和点均没有M可扩路才停止。

\textbf{偶图寻找最大匹配算法}
设M是$G=(X, Y)$的初始匹配。
(1) 置$S=\Phi, T=\Phi$;
(2) 若$X-S$已经M饱和,停止;否则,设u是$X-S$中的一非饱和顶点,置$S=S \cup\{u\}$。
(3) 若$N(S)=T$,转(5);否则,设$y \in N(S)-T$。
(4) 若y是M饱和的,设$yz \in M$,置$ S=S\cup\{z\}, T=T \cup\{y\}$,转(3); 否则,存在$(u, y)$交错路是M可扩路P,置$M=M\Delta E(P)$,转(1).
(5) 若$X-S=\Phi$,停止;否则转(2)。

\subsection{最优匹配算法}

问题:设$G=(X, Y)$是边赋权完全偶图,且$X=\{x_1, x_2,…,x_n\}$
$Y=\{y_1, y_2,…,y_n\}, w_{ij}=w(x_iy_j)$。在G中求出一个具有最大权
值的完美匹配。由于$K_{n,n}$有$n!$个不同完美匹配,所以枚举计算量是$n!$。

\textbf{定义2 可行顶点标号}:设$G = \{X,Y\}$,对于任意的$x \in X, y \in Y$,有:$l(x)+l(y) \ge w(x,y)$  称 $l$ 是赋权完全偶图G的可行顶点标号。
对于任意的赋权完全偶图G,均存在G的可行顶点标号。
事实上,以下一种设计就是G的一个可行定点标号。
$$
l(x) = max_{y \in Y}(w(xy)), \text{若} x \in X, l(y) = 0, \text{若} y \in Y
$$
\textbf{定义3 相等子图} 设$l$ 是赋权完全偶图$G=(X, Y)$可行顶点标号,令:
$$ E_l = \{xy \in E(G) | l(x) + l(y) = w(x,y)\}$$称$G_l = G [E_l]$为G的对应于$l$的相等子图。

\textbf{定理} 设 $l$ 是\textbf{赋权}完全偶图$G=(X, Y)$的可行顶点标号,若相等子图$G_l$有完美匹配$M*$,则$M*$是$G$的最优匹配。

根据上面定理,如果找到一种恰当可行顶点标号,使得对应的相等子图有完美匹配M*,则求出了G的最优匹配。

\textbf{Kuhn算法}:采用顶点标号修改策略

(1) 若X是M饱和的,则M是最优匹配。否则,令u是一个M非饱和点,置$ S = \{u\}, T = {\Phi}$
(2) 若$T \subset N_{G_l}(S)$,转(3)。否则,计算:
$$\alpha_l = min_{x\in S, y \notin T}\{l(x)+l(y)-w(xy)\}$$
$$ \hat{l} = l(v)-\alpha_l, v \in S;   l(v)+\alpha_l, v \in T;   l(v), else$$
给出新的可行顶点标号,在新标号下重新开始。
(3) 在$N_{G_l}(S)/T$中选择点y。若y是M饱和的, $yz \in M$,则置$S = S \cup \{z\}, T = T \cup \{y\}$ 转(2); 否则,设P是$G_l$中M可扩路,置$M=M\Delta E(P)$,转(1)。

\section{图的着色}

\subsection{图的边着色}

\subsubsection{相关概念}

定义1 \textbf{k边可着色}:设G是图,对G的边进行染色,若\textbf{相邻边染不同的颜色},则称对G进行\textbf{正常边着色}。如果能用k种颜色对G进行正常边着色,称G是\textbf{k边可着色}的

定义2 \textbf{边色数}:设G是图,对G进行正常边着色所需要的最少颜色数,成为G的边色数。

\subsubsection{几类特殊图的边色数}

\textbf{偶图的边色数}

定理1 偶图的边色数:$\chi'(K_{m,n}) = \Delta$

定义3 缺色:设$\pi$是G的一种正常边着色,若点u关联的边的着色没有用到色$i$,则称点u缺i色。

定理2 哥尼定理:若G是偶图,则$\chi'(G) = \Delta$。($\Delta$最大度)(证明,归纳法)

\textbf{一般简单图边着色}

引理:设G是简单图, $x$与$y_1$是G中不相邻的两个顶点,$\pi$是G的一个正常k边着色。若对该着色$\pi$, $x,y_1$以及与$x$相邻点均至少缺少一种颜色,则$G+xy_1$是$k$边可着色的。

定理3 维津定理:若G是单图,则$\chi'(G) = \Delta \text{或者} \Delta + 1$

\textbf{三类特殊简单图的边色数}

定理4 简单图G且最大度大于0,若G只有一个最大度顶点或者两个相邻的最大度顶点,则$\chi'(G) = \Delta$。

定理5 设G是单图,若点数$n = 2k+1$且边数$m > k\Delta$,则:$\chi'(G) = \Delta + 1$。

定理6 设G是奇数阶$\Delta$正则单图,若$\Delta > 0$,则:$\chi'(G) = \Delta + 1$。

定理7 设无环图G中边数最大的重数为$\mu$,则$\chi'(G) \le \Delta + \mu$。

\subsubsection{边着色的应用}

分配课时问题,比赛安排问题,看课件,感觉考试靠的就是这种。

\subsection{图的点着色}

\subsubsection{相关概念}

定义1 \textbf{正常顶点着色,点色数}:设G是一个图,对G的每个顶点着色,使得\textbf{相邻顶点着不同颜色},称为对G的正常顶点着色;如果用k种颜色可以对G进行正常顶点着色,称G可k正常顶点着色;对图G正常顶点着色需要的最少颜色数,称为图G的点色数,图G的点色数用$\chi(G)$ 表示。(而边色数是$\chi'$)

定义2 \textbf{k色图}:色数为k的图称为k色图。

\subsubsection{点色数几个结论}

定理1:对任意G,有$\chi(G) \le \Delta(G) + 1$。

算法:图G的$\Delta(G) + 1$的正常点着色算法,该算法不能算出点色数。

定理2 (布鲁克斯,1941) 若G是连通的单图,并且它既不是奇圈,又不是完全图,则有$\chi(G) \le \Delta(G)$。

定义3 次大度:设G是至少有一条边的简单图,定义:$$\Delta_2(G) = \mathop{max}\limits_{u \in V(G)} \mathop{max}\limits_{y \in N(u), d(v) \le d(u)}  d(v)$$其中$N(u)$是G中点u的邻域,称$\Delta_2$为G的次大度。

计算次大度方法:令$V_2(G) = \{v|v\in V(G), N(v)\text{中存在点u,满足}d(u) \ge d(v)\}$,那么$\Delta_2(G) = max\{d(v) | v \in V_2(G)\}$。

定理3:设G是一个非空简单图,则:$\chi(G) \le \Delta_2(G) + 1$。

推论:设G是非空简单图,若G中最大度的点互不邻接,则有$\chi(G) \le \Delta(G)$。

\subsubsection{四色定理与五色定理}
	
定理4 希伍德 每个平面图是5可着色的。

\subsubsection{顶点着色的应用}

课程安排问题

\end{document}